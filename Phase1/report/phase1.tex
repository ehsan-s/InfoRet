\documentclass[11pt]{article}
\usepackage{url}
%You should edit TheoryOfComputation-lecture00.tex, not this file!
\usepackage{fullpage}
\usepackage{color}
\usepackage{graphicx}
\usepackage{epsfig}
\usepackage{amsthm}
\usepackage{latexsym}
\usepackage{amssymb}
\usepackage{amsmath}
\usepackage{algorithmicx,algorithm}
\usepackage[noend]{algpseudocode}
\usepackage{verbatim}
\usepackage{mathrsfs}
%\usepackage{eso-pic}
\usepackage{unicode-math}
%\setmathfont{XITS Math}
\usepackage{xepersian}
\settextfont{XBZar}
%\setdigitfont{ParsiDigits}
%\defpersianfont\outline[Scale=1]{XBZar}


\newcommand{\newfontobj}[2]{
  \newcommand{#1}[1]{
    \expandafter\def\csname##1\endcsname{{#2 ##1}}}}

\newfontobj{\class}{\rm} % Typeset Classes in roman font

% Some standard classes (use in only mathmode)
% Usage example: $\P \subseteq \NP$ and we believe that $\NP$ is not 
%  equal to $\P$.


\class{PSPACE}	
\class{L}
\class{BPL}
\class{RL}
\class{NC}
\class{ZPL}
\class{NPSPACE}	
\class{ASPACE}	
\class{NL}
\class{EXP}
\class{NEXP}
\class{coNEXP}
\class{NE}
\class{E}
\class{AM}		
\class{MA}
\class{NP}
\class{DNP}
\class{UP}
\class{P}
\class{RP}
\class{BPP}
\class{ZPP}
\class{EXPSPACE}
\class{coNP}
\class{coRP}
\class{coAM}
\class{PH}
\class{IP}
\class{PCP}
\class{MIP}

% operator classes.
\class{BP}

% these commands should be used in math mode - $ $
\newcommand{\SHARPP}{{\#\rm{P}}}
\newcommand{\PARITYP}{{\oplus\rm{P}}}

% math operators...
\DeclareMathOperator{\poly}{poly}
\DeclareMathOperator{\Majority}{Majority}
\DeclareMathOperator{\quasipoly}{quasi-poly}
\DeclareMathOperator{\polylog}{poly-log}
\DeclareMathOperator{\superpoly}{super-poly}
\DeclareMathOperator{\DTISP}{DTISP}
\DeclareMathOperator{\DSPACE}{DSPACE}
\DeclareMathOperator{\DTIME}{DTIME}
\DeclareMathOperator{\NSPACE}{NSPACE}
\DeclareMathOperator{\NTIME}{NTIME}
\DeclareMathOperator{\BPTIME}{BPTIME}
\DeclareMathOperator{\RTIME}{RTIME}
\DeclareMathOperator{\ZPTIME}{ZPTIME}
\DeclareMathOperator{\BPSPACE}{BPSPACE}
\DeclareMathOperator{\RSPACE}{RSPACE}
\DeclareMathOperator{\ZPSPACE}{ZPSPACE}


% Complexity class
\newcommand{\CC}{\mathcal{C}}

%------------------------ Algorithm ------------------------------------

\newenvironment{الگوریتم}[1]
{\bigskip\bigskip\begin{algorithm}\caption{#1} \label{الگوریتم: #1}\vspace{0.5em}\begin{algorithmic}[1]}
		{\end{algorithmic}\vspace{0.5em}\end{algorithm}\bigskip}


\renewcommand{\algorithmicfor}{{به ازای}}
\renewcommand{\algorithmicwhile}{{تا وقتی}}
\renewcommand{\algorithmicdo}{\hspace{-.2em}:}
\renewcommand{\algorithmicif}{{اگر}}
\renewcommand{\algorithmicthen}{\hspace{-.2em}:}
\renewcommand{\algorithmicelse}{{در غیر این صورت:}}
%\renewcommand{\algorithmicelsif}{{در غیر این صورت اگر: }}
\renewcommand{\algorithmicreturn}{{برگردان}}
\renewcommand{\algorithmiccomment}[1]{$\triangleleft$ \emph{#1}}
\renewcommand{\algorithmicrequire}{\textbf{ورودی:}}
\renewcommand{\algorithmicensure}{\textbf{خروجی:}}

\newcommand{\اگر}{\If}
\newcommand{\وگرنه}{\Else}
\newcommand{\وگر}{\ElsIf}
%\newcommand{\پایان‌اگر}{\EndIf}
\newcommand{\بهه}{\For}
%\newcommand{\پایان‌به‌ازای}{\EndFor}
\newcommand{\تاوقتی}{\While}
%\newcommand{\پایان‌تاوقتی}{\EndWhile}
\newcommand{\دستور}{\State}
\newcommand{\دستورک}{\Statex}
\newcommand{\توضیحات}{\Comment}
\newcommand{\برگردان}{\Return}
\newcommand{\ورودی}{\Require}
\newcommand{\خروجی}{\Ensure}



%---------------------------------------------------------------------

% \lecture{number}{date}{title}{scribe}
\newcommand{\lecture}[2]{
\noindent
\fbox{
\begin{minipage}{6.2in}
 {\bf بازیابی پیشرفته‌ی اطلاعات}  \hfill 	\includegraphics[scale=0.08]{Sharif.png}\hfill  نیم‌سال اول ۱۳۹۹-۱۳۹۸
  \begin{center}
    {\Large گزارش فاز #1 پروژه} \\[3mm]
  \end{center}
مدرس: دکتر بیگی\hfill اعضای گروه: #2
\end{minipage}
}
\bigskip

\bigskip
}
% \homework{number}{date}
\newcommand{\homework}[2]{
\noindent
\fbox{
\begin{minipage}{6.2in}
  {\bf CS 810: Complexity Theory} \hfill #2
  \begin{center}
    {\Large Homework #1} \\[3mm]
  \end{center}
Instructor: Dieter van Melkebeek \hfill TA: Jeff Kinne
\end{minipage}
}
\bigskip

\bigskip
}

% add DRAFT to your document %
\newcommand{\draft}[0]{
\begin{center}
	{\bf \LARGE {\sc نسخه اولیه} }
\end{center}
}

% example environment
\newenvironment{example}
{\smallskip \noindent \emph{مثال:}}
{\hfill $\boxtimes$ \smallskip}

% some theorem environments
\newtheorem{theorem}{قضیه}
\newtheorem{conjecture}[theorem]{حدس}
\newtheorem{proposition}{گزاره}
\newtheorem{claim}{ادعا}
\newtheorem{lemma}{لم}
\newtheorem{corollary}{نتیجه}
\newtheorem{definition}{تعریف} % Use this for non-trivial definitions.

% currently not used %
\newtheorem{exercise}{تمرین}
\newtheoremstyle{example}{\topsep}{\topsep}%
     {\normalfont \small}   % Body font
     {}    % Indent amount (empty = no indent, \parindent = para indent)
     {\bfseries}     % Thm head font
     {}%           Punctuation after thm head
     {\topsep}%     Space after thm head
     {}%         Thm head spec    \theoremstyle{example}
\theoremstyle{example}
%\newtheorem{example}{Example}



\begin{document}
\lecture{اول}%
{احسان سلطان‌آقایی، وحید بالازاده مرشت}

\section*{مشارکت وحید بالازاده}
درصد مشارکت: ۵۰ درصد.\\

کارهای انجام‌گرفته:
\begin{itemize}
\item بخش ۲: کد این بخش در فایل \lr{indexer.py} و در کلاس \lr{Indexer} قابل مشاهده است. در این بخش با استفاده از متودهای \lr{add\_doc} و \lr{del\_doc} می‌توان سندی را به نمایه اضافه و یا حذف کرد. از فیلدهای این کلاس در قسمت جستجو و بازیابی نیز استفاده می‌شود.
\item بخش ۵: کد این بخش در فایل \lr{search.py} و در کلاس \lr{Searcher} قرار دارد. دو متد \lr{search} و \lr{search\_prox} به ترتیب برای جستجوی عادی و جستجوی \lr{proximity} استفاده می‌شوند. در هر دو متود روش \lr{lnc.ltc} به کار رفته‌است.
\item تست کلاس‌های \lr{Searcher} و \lr{Indexer} نیز در فایل \lr{test\_search.py} آمده‌است.
\item کد مربوط به بخش کنسول که در فایل \lr{main.py} قرار دارد.
\end{itemize}
\section*{مشارکت احسان سلطان‌آقایی}
درصد مشارکت: ۵۰ درصد.\\

کارهای انجام‌گرفته:
\begin{itemize}
\item 
بخش ۱: کد این بخش در فولدر \lr{preprocess} موجود است. در این قسمت داده‌های فارسی به فرمت \lr{xml} خوانده می‌شوند. از مجموعه داده‌های فارسی قسمت عنوان و متن صفحه‌های ویکی‌پدیا تحت یک متن به پیش‌پردازشگر داده می‌شود. از مجموعه داده‌های انگلیسی نیز عنوان و متن اخبار تحت یک متن به پیش‌پردازشگر داده می‌شود. پیش‌پردازشگر انگلیسی از کتابخانه \lr{NLTK} و پیش‌پردازشگر فارسی از کتابخانه \lr{hazm} استفاده می‌کند. هم‌چنین درصد معقولی از کلمات پرتکرار با پردازش متن و به کمک نمایش آن حذف می‌شوند.
\item 
بخش ۳: کد این بخش در فولدر \lr{compression} قرار دارد. نمایه ساخته‌شده در بخش ۲ را دریافت می‌کند و به دو روش \lr{variable byte} و \lr{gamma code} فشرده‌سازی می‌شود. نتیجه میزان حافظه اشغال شده به این صورت است که ذخیره سازی به صورت عادی ۱۴ مگابایت، ذخیره‌سازی به روش \lr{variable byte} حدود ۶ مگابایت و ذخیره‌سازی به روش \lr{gomma code} حدود ۷ مگابایت فضا اشغال می‌کند.
\item
بخش ۴: کد این بخش در فولدر \lr{edit query} موجود است. یک پرسمان دریافت می‌کند. ابتدا تشخیص می‌دهد که فارسی است یا انگلیسی، سپس پیش‌پردازش متناسب را روی آن انجام می‌دهد. سپس کلمات پرسمان را به ترتیب با کلمات نمایه به روش \lr{bigram} و با معیار \lr{jaccard} مقایسه می‌کند. در نهایت نزدیک ترین کلمه از بین کلمات منتخب به روش ذکر شده را با معیار \lr{edit distance} جایگزین کلمه پرسمان می‌کند.


\end{itemize}

\end{document}
